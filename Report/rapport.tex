\documentclass{article}
\usepackage[utf8]{inputenc}
\usepackage{amsmath}
\usepackage{amssymb}
\usepackage{graphicx}
\usepackage{algorithm}
\usepackage[noend]{algpseudocode}
\usepackage[hidelinks]{hyperref}

\title{Problème de clique maximum}

\author{Drissi Mohamed Reda}

\begin{document}

\maketitle
\newpage
\tableofcontents
\newpage
\section{Introduction}
Nous prenons un graphe $G=\{V,E\}$ tel que $V$ est l'ensemble des sommets
et $E$ l'ensemble des arrêtes(Nous ignorons les arrêtes liant une arrête à elle même). \\
Nous nommons les sommets du graphe 1,2, ... n. Cet algorithme permet de trouver une clique
de taille d'au moins $k$. À chaque étape, si nous trouvons une clique de taille d'au moins
$k$, nous arrêtons.
\section{Algorithme et Test}
\subsection{Algorithme}
\begin{algorithm}
\begin{algorithmic}
  \Function{Expand\_clique}{Graph Q}
  \While{$\exists$ v joignable à Q}
    \State trouver $v_{max}$ tel que Le nombre de sommets joignable à la clique $Q \cup \{v_{max}\}$ est maximal
    \State $Q \gets Q\cup \{v_{max}\}$
  \EndWhile
    \State \Return $Q$
  \EndFunction
\end{algorithmic}
\end{algorithm}
\begin{algorithm}
\begin{algorithmic}
  \Function{Alter\_Clique}{Graph Q}
  \State Trouver $v \notin Q$ tel que $\exists! w \in Q$ non adjacent à $v$.
    \If{$v = \varnothing$}
      \State \Return $Q$
    \Else
      \State $Q\gets Q \setminus \{w\}$
      \State $Q \gets Q \cup \{v\}$
      \State $Q \gets$ \Call{Expand\_clique}{Q}
      \State \Return $Q$
    \EndIf
  \EndFunction
\end{algorithmic}
\end{algorithm}

\begin{algorithm}[!htp]
\caption{Algorithme de clique de taille k}
\begin{algorithmic}[1]
  \For{i $\in$ [1,n]}
    \State initialiser la clique $Q_i\gets \{i\}$
    \State $Q_i \gets$ \Call{Expand\_Clique}{$Q_i$}
    \For{i $\in$ [1,k]}
      \State $Q_i \gets$ \Call{Alter\_Clique}{$Q_i$}
    \EndFor
  \EndFor
  \ForAll{$Q_i$ et $Q_j$ trouvés}
    \State initialiser la clique $Q_{i,j} \gets Q_i \cap Q_j$
    \State $Q_{i,j} \gets$ \Call{Expand\_Clique}{$Q_{i,j}$}
    \For{i $\in$ [1,k]}
      \State $Q_i \gets$ \Call{Alter\_Clique}{$Q_i$}
    \EndFor
  \EndFor

\end{algorithmic}
\end{algorithm}
\subsection{Example}
Comme example nous prenons le graphe complémentaire du \href{https://en.wikipedia.org/wiki/Frucht\_graph}{graphe de Frucht}
Avec les sommets :
\begin{displaymath}
  V=\{1,2,3,...12\}
\end{displaymath}
\begin{center}
  \begin{figure}[!htp]
    \caption{Complément du graphe de Frucht}
    \includegraphics[scale=0.7]{Report/frucht-compl.png}
  \end{figure}
\end{center}

Nous allons faire tourner cet algorithme pour trouver une clique d'au moins $k=5$. \\
Pour i=1 et i=2 nous trouvons des cliques de taille 4 donc nous passons à i=3.
Initialisation : $Q_3=\{i\}=\{3\}$. \\
Nous exécutons la fonction \textit{Expand\_Clique}
\begin{center}
  \begin{tabular}{|c|c|c|}
    \hline
    sommet $v$ joignable à $Q_3$ & sommets joignables à $Q_3\cup\{v\}$ & taille \\ \hline
    \hline
    1 	& 5, 6, 8, 9, 12  	& 5 \\ \hline
    5   & 1, 7, 8, 11, 12   & 5 \\ \hline
    6 	& 1, 9, 11, 12  	  & 4 \\ \hline
    7 	& 5, 9, 11, 12  	  & 4 \\ \hline
    8 	& 1, 5, 9, 11 	    & 4 \\ \hline
    9 	& 1, 6, 7, 8, 11 	  & 5 \\ \hline
    11 	& 5, 6, 7, 8, 9, 12 & 6 \\ \hline
    12 	& 1, 5, 6, 7, 11  	& 5 \\ \hline
  \end{tabular}
\end{center}
Nous trouvons que le maximum est :
\begin{displaymath}
  |Q_3 \cup \{v\}|_{max}=6 \quad pour \quad v=11
\end{displaymath}
Nous ajoutons le sommet $6$ à $Q$. Nous retrouvons la nouvelle clique \\
$Q_3=\{3, 11\}$ de taille 2.
\begin{center}
  \begin{tabular}{|c|c|c|}
    \hline
    sommet $v$ joignable à $Q_3$ & sommets joignables à $Q_3\cup\{v\}$ & taille \\ \hline
    \hline
    5 	& 7, 8, 12 	& 3 \\ \hline
    6 	& 9, 12 	  & 2 \\ \hline
    7 	& 5, 9, 12 	& 3 \\ \hline
    8 	& 5, 9 	    & 2 \\ \hline
    9 	& 6, 7, 8 	& 3 \\ \hline
    12 	& 5, 6, 7 	& 3 \\ \hline
  \end{tabular}
\end{center}
Nous trouvons que le maximum est :
\begin{displaymath}
  |Q_3 \cup \{v\}|_{max}=3 \quad pour \quad v=5
\end{displaymath}
Nous ajoutons le sommet $5$ à $Q$. Nous retrouvons la nouvelle clique \\
$Q_3=\{3, 5, 11\}$ de taille 3.
\begin{center}
  \begin{tabular}{|c|c|c|}
    \hline
    sommet $v$ joignable à $Q_3$ & sommets joignables à $Q_3\cup\{v\}$ & taille \\ \hline
    \hline
    7 	& 12 	            & 1 \\ \hline
    8 	& $ \varnothing$ 	& 0 \\ \hline
    12 	& 7 	            & 1 \\ \hline
  \end{tabular}
\end{center}
Nous trouvons que le maximum est :
\begin{displaymath}
  |Q_3 \cup \{v\}|_{max}=1 \quad pour \quad v=7
\end{displaymath}
Nous ajoutons le sommet $12$ à $Q$. Nous retrouvons la nouvelle clique \\
$Q_3=\{3, 5, 7, 11\}$ de taille 4.
\begin{center}
  \begin{tabular}{|c|c|c|}
    \hline
    sommet $v$ joignable à $Q_3$ & sommets joignables à $Q_3\cup\{v\}$ & taille \\ \hline
    \hline
    12    & $\varnothing$ & 0 \\ \hline
  \end{tabular}
\end{center}
Nous trouvons que le maximum est :
\begin{displaymath}
  |Q_3 \cup \{v\}|_{max}=0 \quad pour \quad v=12
\end{displaymath}
Nous ajoutons le sommet $12$ à $Q$. Nous retrouvons la nouvelle clique \\
$Q_3=\{3, 5, 7, 11, 12\}$ de taille 5.\\
Puisque la clique est de taille 5, nous nous arrêtons là.
\textbf{FIX THE VERTICES NAMING}
\section{Implémentation}
Nous implémentons en C++ l'algorithme trouvé précédemment pour trouver la clique
maximum d'un graphe.

\section{Test}
Nous allons faire tourner notre code C++ sur plusieurs graphes, notre algorithme
nous permet de choisir la taille de la clique souhaitée, cela rends le code plus partique
pour s'assurer de trouver la clique maximale, nous allons choisir un nombre n=k
Ce graphe :
\begin{center}
  \begin{verbatim}
    0 0 0 1 0 0 1
    0 0 0 0 1 0 1
    0 0 0 0 0 1 1
    1 0 0 0 1 1 0
    0 1 0 1 0 1 0
    0 0 1 1 1 0 0
    1 1 1 0 0 0 0
  \end{verbatim}
\end{center}
donne le résultat :
\begin{center}
  \begin{verbatim}
  1. Clique of size 3 : 4 5 6
  \end{verbatim}
\end{center}
\end{document}
